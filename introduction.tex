%%%%%%%%%%%%%%%%%%%%%%%%%%%%%%%%%%%%%%%%%%%%%%%%%%%%%%%%%%%%%%%%%
\section{Introduction}
\label{sec:introduction}
%%%%%%%%%%%%%%%%%%%%%%%%%%%%%%%%%%%%%%%%%%%%%%%%%%%%%%%%%%%%%%%%%

Today's end-user mobile devices have become indispensable gadgets in people's 
everyday life. A greater variety of online activities are increasingly conducted on 
mobile devices. As a result, the Internet architecture, end-user 
traffic patterns, etc., have evolved rapidly over the past few years. To understand 
the complex, underlying structure of mobile Internet, and 
to study how new applications and services can perform better, network researchers 
need to collect and study datasets from end-user mobile devices. 

Mobile devices %have become indispensable gadgets in people's everyday life. 
are equipped with sensors\footnote{In this work, sensors are broadly defined 
as the hardware components that can record phenomena about the physical 
world, such as WiFi/cellular network, GPS location, movement acceleration, etc.} 
that can provide useful dataset for research. 
%Network researchers can perform experiments by 
%collect sensor data from these end-user mobile devices to study 
%how new applications and services can provide better performance. 
For example, 
wireless researchers need to know the coverage of a cellular service in an urban 
district compared to a rural area, whether there exist any service blind spots
with weak cellular signal strength, and so on. The data collected
can generate useful information for service providers, policy 
makers and researchers to enhance network research and the service provided.

However, privacy and security issues have traditionally been the 
obstacles to sharing sensor data. First, sensor data can reveal device 
owners' personal information and result in privacy concerns. Second, potential 
bugs in the research experiment can damage end users' personal devices
and cause security issues. 
Due to these privacy and security risks, researchers often choose to recruit 
participants within a trusted group, e.g., students, colleagues and friends. 
The data collected cannot represent real-world scenarios, 
and the researchers are not able to test their hypothesis at a large scale. 

To address these challenges, we 
design and implement \sysname~\cite{sensibility}, an Internet-wide 
measurement testbed for mobile devices. \sysname enables network 
researchers to unobtrusively collect data 
from the sensors on the device, such as GPS, Bluetooth, battery information, 
accelerometer, light, and orientation, etc. More importantly, it
mediates the process for device owners to make their devices 
available while preserving their privacy, and enables researchers to 
securely experiment with end user mobile device and data. 
This work is a first step towards protecting the security and privacy of end-user 
devices, and lowering the technical barriers to network research done 
through wide-area sensor data collection. 